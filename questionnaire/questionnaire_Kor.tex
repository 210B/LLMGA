\documentclass[12pt]{article}
\usepackage[utf8]{inputenc}
\usepackage[a4paper, margin=1in]{geometry}
\usepackage{enumitem}
\usepackage{titlesec}
\usepackage{hyperref}
\usepackage{kotex}

\titleformat{\section}{\large\bfseries}{\thesection.}{1em}{}

\title{\textbf{Game Experience Questionnaire}\\[0.5em]
\large A Study Conducted by the Human-Computer Interaction Research Lab @HCIL}

\author{Department of Artificial Intelligence and Software\\
Ewha Womans University\\
Hayeon Kim, Jisu Kim, Soyeon Kim\\
\texttt{hayy@ewha.ac.kr}}

\begin{document}

\maketitle
\hrule
\vspace{0.5cm}

\section*{응답 방법}
아래의 각 문항에 대해 본인의 생각과 가장 잘 맞는 번호에 동그라미를 쳐 주세요.  
1은 ‘전혀 그렇지 않다’, 5는 ‘매우 그렇다’를 의미합니다.

\vspace{0.5cm}

\begin{center}
\begin{tabular}{|c|c|}
\hline
1 & 전혀 그렇지 않다 \\
\hline
2 & 그렇지 않다 \\
\hline
3 & 보통이다 \\
\hline
4 & 그렇다 \\
\hline
5 & 매우 그렇다 \\
\hline
\end{tabular}
\end{center}

\vspace{0.5cm}
\hrule
\vspace{0.5cm}

\section*{1. 주의 집중}

\begin{enumerate}[label=\arabic*.]
  \item 게임이 내 주의를 끌었다.
  \item 나는 게임에 집중하고 있었다.
  \item 나는 게임을 하기 위해 노력을 기울였다.
  \item 나는 최선을 다하고 있다고 느꼈다.
\end{enumerate}

\section*{2. 시간 감각의 소실}

\begin{enumerate}[resume]
  \item 게임을 하면서 시간 가는 줄 몰랐다.
  \item 게임 중에도 현실 세계를 의식하고 있었다. \textit{(역채점)}
  \item 일상적인 걱정을 잊고 있었다.
  \item 주변의 나 자신을 인식하고 있었다. \textit{(역채점)}
  \item 주변에서 일어나는 일을 인지하고 있었다. \textit{(역채점)}
  \item 주위를 살피기 위해 게임을 멈추고 싶은 충동이 들었다. \textit{(역채점)}
\end{enumerate}

\section*{3. 몰입감 (Transportation)}

\begin{enumerate}[resume]
  \item 현실 환경과 분리된 느낌이 들었다.
  \item 게임은 활동이라기보다는 하나의 경험처럼 느껴졌다.
  \item 게임 속 존재감이 현실보다 더 강하게 느껴졌다.
  \item 게임에 너무 몰입해서 조작하고 있다는 것을 잊었다.
  \item 내가 원하는 대로 게임 속에서 움직이고 있다는 느낌이 들었다.
\end{enumerate}

\section*{4. 감정적 몰입}

\begin{enumerate}[resume]
  \item 게임에 감정적으로 몰입되었다.
  \item 게임의 전개가 어떻게 될지 궁금했다.
  \item 시뮬레이션의 끝이나 최종 상태에 도달하는 데 관심이 있었다.
  \item 게임에 너무 몰입해서 게임과 직접 대화하고 싶었다.
\end{enumerate}

\section*{5. 즐거움}

\begin{enumerate}[resume]
  \item 게임 속 대화(채팅)를 즐겼다.
  \item 게임하는 것이 즐거웠다.
  \item 이 게임을 다시 하고 싶다.
\end{enumerate}

\section*{6. 캐릭터 일관성}

\begin{enumerate}[resume]
    \item 캐릭터가 자신의 성격이나 말투와 맞지 않는 말을 했다. \textit{(역채점)}
    \item 캐릭터가 일관된 성격을 유지 했다.
    \item 캐릭터가 지나치게 차분하거나 논리적이거나 과장된 방식으로 행동해 어색했다. \textit{(역채점)}
    \item 캐릭터가 내 행동이나 대사의 의도를 적절하게 이해했다.
    \item 캐릭터의 행동이나 감정이 이전 상황과 일치하지 않았다. \textit{(역채점)}
    \item 부조리하거나 일관성 없는 캐릭터의 행동이 너무 자주 등장해 몰입이 방해되었다. \textit{(역채점)}
    \item 캐릭터의 대사가 감정적 몰입(공감, 긴장 등)을 방해했다. \textit{(역채점)}
    \item 특정 대사 이후, 게임의 다음 전개에 대한 흥미가 떨어졌다. \textit{(역채점)}
\end{enumerate}

\end{document}
